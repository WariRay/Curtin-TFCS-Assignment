%%%%%%%%%%%%%%%%%%%%%%%%%%%%%%%%%%%%%%%%%
% Lachaise Assignment
% LaTeX Template
% Version 1.0 (26/6/2018)
%
% This template originates from:
% http://www.LaTeXTemplates.com
%
% Authors:
% Marion Lachaise & François Févotte
% Vel (vel@LaTeXTemplates.com)
%
% License:
% CC BY-NC-SA 3.0 (http://creativecommons.org/licenses/by-nc-sa/3.0/)
% 
%%%%%%%%%%%%%%%%%%%%%%%%%%%%%%%%%%%%%%%%%

%----------------------------------------------------------------------------------------
%	PACKAGES AND OTHER DOCUMENT CONFIGURATIONS
%----------------------------------------------------------------------------------------

\documentclass{article}

\input{structure.tex} % Include the file specifying the document structure and custom commands

\usepackage[
style=ieee,
backend=biber
]{biblatex}
\addbibresource{references.bib}


%----------------------------------------------------------------------------------------
%	ASSIGNMENT INFORMATION
%----------------------------------------------------------------------------------------

\title{
    TFCS Group 7 Assignment \\  
    \large{ % Subtitle
		\vspace{0.3cm}
        Beam Me Down, Scotty
    }
} % Title of the assignment

\author{Moritz Bergemann (19759948), Kristian Rados (19764285), Warittha Rayabsri (19747859)} % Author name and email address

\date{\today}

%----------------------------------------------------------------------------------------

\begin{document}

\maketitle % Print the title

%----------------------------------------------------------------------------------------
%	INTRODUCTION
%----------------------------------------------------------------------------------------

\section{Executive Summary} % Unnumbered section

Lorem ipsum dolor sit amet, consectetur adipiscing elit. Praesent porttitor arcu luctus, imperdiet urna iaculis, mattis eros. Pellentesque iaculis odio vel nisl ullamcorper, nec faucibus ipsum molestie. Sed dictum nisl non aliquet porttitor. Etiam vulputate arcu dignissim, finibus sem et, viverra nisl. Aenean luctus congue massa, ut laoreet metus ornare in. Nunc fermentum nisi imperdiet lectus tincidunt vestibulum at ac elit. Nulla mattis nisl eu malesuada suscipit.

% Math equation/formula
\begin{equation}
	I = \int_{a}^{b} f(x) \; \text{d}x.
\end{equation}

Aliquam arcu turpis, ultrices sed luctus ac, vehicula id metus. Morbi eu feugiat velit, et tempus augue. Proin ac mattis tortor. Donec tincidunt, ante rhoncus luctus semper, arcu lorem lobortis justo, nec convallis ante quam quis lectus. Aenean tincidunt sodales massa, et hendrerit tellus mattis ac. Sed non pretium nibh. Donec cursus maximus luctus. Vivamus lobortis eros et massa porta porttitor.

%----------------------------------------------------------------------------------------
%	QUESTION DEFINITION
%----------------------------------------------------------------------------------------
\section{Question Definition}

\subsection{Provided Question}
The following is the question as detailed in the specification. Some details irrelevant to the problem have been removed.

\begin{question}[Question \#14 - Beam Me Down Scotty]
	Write some software to find the best way to link some communication ground stations. \\
	The ground stations must communicate with each other. There is a limited budget, so communication must be done as cheaply as possible. \\
	You will be given a graph to hide the actual locations (vertices) and communication channels (edges) but you will be given a scaled version of the costs for each channel (edge). \\
	In essence, there needs to be a path of communication methods between any pair of locations and the overall costs of the communication channels should be minimized. \\
\end{question}

\subsection{Conversion to a decision problem}

An analysis of the question reveals the following base requirements:

\begin{itemize}
	\item Input is given in the form of a weighted graph (vertices plus with edges that have costs).
	\item The algorithm must choose a subset of edges to “build” the network (a subset of edges will be selected).
	\item The selected subset of edges must connect each pair of vertices in the graph through some sequence of edges.
\end{itemize}

These requirements match the definition of an existing problem - the identification of \textbf{minimum spanning trees} (MSTs) \cite{sedgewick_minimum_2018}.

Using this intuition, we can simplify the problem further:

 % TODO

%----------------------------------------------------------------------------------------
%	SOLUTIONS TO THE PROBLEM
%----------------------------------------------------------------------------------------
\section{Solutions to The Problem}

%----------------------------------------------------------------------------------------
%	TURING MACHINE
%----------------------------------------------------------------------------------------
\section{Turing Machine Solution}
\section{Turing Machine Implementation}

The following is a Turing Machine implementation of Prim's algorithm.

Notes for understanding the TM:
\begin{itemize}
	\item 7. and 8. effectively check whether an edge would add a new (unvisited) vertex to the MST and is connected to the current MST. In other words, one of the edge's vertices must be visited, but not both.
	\item This turing machine is functional with an arbitrary number of vertices and edges, and arbitrary (integer) edge weights.
\end{itemize}

 % Turing machine start
\begin{lstlisting}[frame=single]
The machine contains the following tapes:
- Input tape
- Visited tape 
- Current_Minimum tape
- Minimum_Spanning_Tree tape
- Minimum_Spanning_Tree_Cost tape
- Misc tape
 
 
######################
## INPUT VALIDATION ##
######################
## Input will be in the following format:
## A representation of a tree as a set of vertex names (which 
##     must be alphabetical letters), 
##     followed by a set of weighted edges (where weights are in 
##     binary) followed by an input number k (in binary)
## For example: (A, B, C), ((A, B, 01), (B, C, 10), (A, C, 11)), 110
##     (spacing only for clarity)
# NOTE: We will use the visited tape to keep track of which vertices we
#     have seen already
0. SPECIAL RULES:
    - If we are moving right until we find something and we reach
        the end of the tape, reject unless otherwise specified.
# Checking vertices
1. Check value at Input tape. If not '(', Reject.
2. Move right on Input tape. If value is not alphabetical letter, 
        reject.
4. Move right on Input tape until ',' or ')' found. Reject if anything
        inbetween is not alphabetical letter.
5. Move left on Input tape until ',' or '(' found. Move right. 
        Perform VISIT_VERTEX.
6. Move right on Input tape until ',' or ')' found.
        - If ',' found, go to 2.
        - If ')' found, go to 7.

# Checking edges
7. Move right on Input tape. If value not ',' Reject. Move right. 
        If value not '(', Reject. Move right.
8. If value not '(', Reject. Move right until ',' found. If anything
        inbetween is not alphabetical letter, reject.
9. Move left until '(' found. Perform CHECK_VISITED. If no, Reject.
10. Move right until ',' found.
11. Move right until ',' found. If anything inbetween is not 
        alphabetical letter, Reject.
12. Move left until ',' found. Perform CHECK_VISITED. If no, Reject.
13. Move right until ',' found. Move right.
14. Move right until ')' found. If anything between is not binary, 
        Reject.
15. Move right. 
        - If value is ')', Go to 16.
        - If value is ',', move right. Go to 8.
        - For any other value, reject.

# Check k
16. Move right. If value not ',', Reject.
17. Move right until end of tape reached. 
            If anything between is not binary, reject.
18. Clear all tapes besides Input tape. Move to beginning of Input 
        tape.


##################
## MAIN PROGRAM ##
##################
0. Perform INPUT VALIDATION.
1. Move right on Input tape. Perform VISIT_VERTEX.
2. Write 0 to the minimum spanning tree cost tape.

3. Move to beginning of Input tape. Move right on Input tape until 
    third '(' found. (Move to first edge)
4. Write 'INF' to Current_Minimum tape
5. Move right on Input tape (moving to edge's first vertex name). 
        Perform CHECK_VISITED.
        - If yes, go to 6.
        - If no, go to 7.
6. Move right on Input tape (moving to edge's second vertex name). 
        Perform CHECK_VISITED.
        - If no, perform CHECK_CURRENT_MINIMUM
        - Go to 8.
7. Move right on Input tape (moving to edge's second vertex name). 
        Perform CHECK_VISITED.
        - If yes, perform CHECK_CURRENT_MINIMUM.
        - Go to 8.
8. move right on Input tape until '(' found (next edge). Go to 5.
        - If no '(' found before end of Input tape, go to 9.
9. If Current_Minimum tape contains 'INF', go to 11.
10. Perform ADD_TO_MST. Go to 3.

# Calculate cost of identified minimum spanning tree
11. Move to beginning of Minimum_Spanning_Tree tape. Move to beginning
        of Minimum_Spanning_Tree_Cost tape.
12. Move right on Minimum_Spanning_Tree tape until '(' found, 
        unless already at '('. Move right on Minimum_Spanning_Tree tape
        until second ',' found, then move right again. If the end of 
        the tape is reached, 
        go to 14.
13. Perform ADDITION(Minimum_Spanning_Tree, Minimum_Spanning_Tree_Cost). 
        Move to beginning of Minimum_Spanning_Tree_Cost tape. Go to 12.

14. Go to beginning of Minimum_Spanning_Tree_Cost tape.
15. Move to end of Input tape, then move left until ',' found, then 
        move right.
16. Perform LESS_THAN(Minimum_Spanning_Tree_Cost, Input).
        - If yes, Accept.
        - If no, Reject.


#################
## SUBROUTINES ##
#################
VISIT_VERTEX:
## Adds the vertex at Input tape to Visited tape.
1. Move to the beginning of Input tape and end of Visited tape.
2. Move right on Visited tape, write ','
3. Move right on Visited tape.
4. Write the value on the Input tape to the Visited tape.
5. Move right on Input tape.
        - If Input tape value is ',', end.
        - Otherwise, go to 4.

CHECK_VISITED:
## Returns yes or no depending on if vertex name at Input tape is in 
    Visited tape.
1. Move to beginning of Visited tape.

# Comparison operation
2. Clear and move to beginning of Misc tape.
3. Check if value at Input tape and Visited tape are the same. # Do 
        once first to read in first value
        - If yes, move right on Input and Visited tape. Write 'X' to
            Misc tape and move right. Go to 4.
        - If no, go to XXX.
4. Check if value at Visited tape is ',' or end of tape. # If end of
        visited label
        - If no, go to 5.
        - If yes, go to 6. 
5. Check if value at Input tape and Visited tape are the same.
        - If yes, move right on Input and Visited tape. Write 'X' to
            Misc tape and move right. Go to 4.
        - If no, go to XXX.
6. If value at Input tape is ',' or ')', Return Yes. Otherwise, go to 7.
# Fail, check next
7. Move left on Input tape for every 'X' on Misc tape. Move right on 
        input tape until ',' found, unless we are already at a ','. 
        - If we reach the end of the tape before finding a ',', 
            Return No.


CHECK_CURRENT_MINIMUM:
## Checks if the cost of the edge Input tape is currently at is less 
##   than cost of edge in Current_Minimum tape. If yes, replaces 
## Current_Minimum tape's value with the one at the Input tape.
1. Move to beginning of Current_Minimum tape.
2. If Current_Minimum contains 'INF', go to XXX
3. Perform LESS_THAN(Input, Current_Minimum)
        - If yes, go to 4.
        - If no, end.
# Make this edge the current minimum edge
4. Move left on Input tape until '(' found.
5. Clear Current_Minimum tape, move to beginning of Current_Minimum 
        tape.
6. Write value at Input tape to Current_Minimum tape. Move right on 
        Input and Current_Minimum tape. 
        - Repeat until Input contains ')'  
7. Move to beginning of Current_Minimum tape.


ADD_TO_MST:
## Adds the vertex to the currently tracked minimum spanning tree, 
##     and visits the unvisited vertex it contained.
1. Move to end of Minimum_Spanning_Tree tape.
2. On Minimum_Spanning_Tree tape: Move right, write ',', move right
3. Write value on Input tape to Minimum_Spanning_Tree tape, move right 
        on both. Repeat until Input tape contains ')', then perform
        write one more time.
4. Move left on Input tape until '(' found. Move right.
5. Perform CHECK_VISITED.
        # If first vertex in edge wasn't visited
        - If no, move left on Input tape until '(' found. Move right.
        # If second vertex in edge wasn't visited
        - If yes, move right on Input tape until ',' found. Move right.
6. Perform VISIT_VERTEX.


ADDITION(Tape_1, Tape_2):
## Adds the binary value starting at Tape_1 to Tape_2, overwriting the
##  value in Tape_2 with the result.
##  If the new value requires more digits for representation, all 
##  values in Tape_2 are shuffled to account for this.
Addition of binary values done as in TODO


LESS_THAN(Tape_1, Tape_2):
## Checks if the binary value at Tape_1 is less than the binary value
##  at Tape_1.
##  Gives Yes or No answer.
##  (Note: Assuming binary values are continued to be read until 
##  non-binary value found on a tape)
Comparison of binary values done as in TODO
\end{lstlisting}

\begin{info} % Information block
	This is an interesting piece of information, to which the reader should pay special attention. Fusce varius orci ac magna dapibus porttitor. In tempor leo a neque bibendum sollicitudin. Nulla pretium fermentum nisi, eget sodales magna facilisis eu. Praesent aliquet nulla ut bibendum lacinia. Donec vel mauris vulputate, commodo ligula ut, egestas orci. Suspendisse commodo odio sed hendrerit lobortis. Donec finibus eros erat, vel ornare enim mattis et.
\end{info}

%----------------------------------------------------------------------------------------
%	PROBLEM 1
%----------------------------------------------------------------------------------------

\section{Problem title} % Numbered section

In hac habitasse platea dictumst. Curabitur mattis elit sit amet justo luctus vestibulum. In hac habitasse platea dictumst. Pellentesque lobortis justo enim, a condimentum massa tempor eu. Ut quis nulla a quam pretium eleifend nec eu nisl. Nam cursus porttitor eros, sed luctus ligula convallis quis. Nam convallis, ligula in auctor euismod, ligula mauris fringilla tellus, et egestas mauris odio eget diam. Praesent sodales in ipsum eu dictum.

%------------------------------------------------

\subsection{Theoretical viewpoint}

Maecenas consectetur metus at tellus finibus condimentum. Proin arcu lectus, ultrices non tincidunt et, tincidunt ut quam. Integer luctus posuere est, non maximus ante dignissim quis. Nunc a cursus erat. Curabitur suscipit nibh in tincidunt sagittis. Nam malesuada vestibulum quam id gravida. Proin ut dapibus velit. Vestibulum eget quam quis ipsum semper convallis. Duis consectetur nibh ac diam dignissim, id condimentum enim dictum. Nam aliquet ligula eu magna pellentesque, nec sagittis leo lobortis. Aenean tincidunt dignissim egestas. Morbi efficitur risus ante, id tincidunt odio pulvinar vitae.

Curabitur tempus hendrerit nulla. Donec faucibus lobortis nibh pharetra sagittis. Sed magna sem, posuere eget sem vitae, finibus consequat libero. Cras aliquet sagittis erat ut semper. Aenean vel enim ipsum. Fusce ut felis at eros sagittis bibendum mollis lobortis libero. Donec laoreet nisl vel risus lacinia elementum non nec lacus. Nullam luctus, nulla volutpat ultricies ultrices, quam massa placerat augue, ut fringilla urna lectus nec nibh. Vestibulum efficitur condimentum orci a semper. Pellentesque ut metus pretium lacus maximus semper. Sed tellus augue, consectetur rhoncus eleifend vel, imperdiet nec turpis. Nulla ligula ante, malesuada quis orci a, ultricies blandit elit.

% Numbered question, with subquestions in an enumerate environment
\begin{question}
	Quisque ullamcorper placerat ipsum. Cras nibh. Morbi vel justo vitae lacus tincidunt ultrices. Lorem ipsum dolor sit amet, consectetuer adipiscing elit.

	% Subquestions numbered with letters
	\begin{enumerate}[(a)]
		\item Do this.
		\item Do that.
		\item Do something else.
	\end{enumerate}
\end{question}
	
%------------------------------------------------

\subsection{Algorithmic issues}

In malesuada ullamcorper urna, sed dapibus diam sollicitudin non. Donec elit odio, accumsan ac nisl a, tempor imperdiet eros. Donec porta tortor eu risus consequat, a pharetra tortor tristique. Morbi sit amet laoreet erat. Morbi et luctus diam, quis porta ipsum. Quisque libero dolor, suscipit id facilisis eget, sodales volutpat dolor. Nullam vulputate interdum aliquam. Mauris id convallis erat, ut vehicula neque. Sed auctor nibh et elit fringilla, nec ultricies dui sollicitudin. Vestibulum vestibulum luctus metus venenatis facilisis. Suspendisse iaculis augue at vehicula ornare. Sed vel eros ut velit fermentum porttitor sed sed massa. Fusce venenatis, metus a rutrum sagittis, enim ex maximus velit, id semper nisi velit eu purus.

\begin{center}
	\begin{minipage}{0.5\linewidth} % Adjust the minipage width to accomodate for the length of algorithm lines
		\begin{algorithm}[H]
			\KwIn{$(a, b)$, two floating-point numbers}  % Algorithm inputs
			\KwResult{$(c, d)$, such that $a+b = c + d$} % Algorithm outputs/results
			\medskip
			\If{$\vert b\vert > \vert a\vert$}{
				exchange $a$ and $b$ \;
			}
			$c \leftarrow a + b$ \;
			$z \leftarrow c - a$ \;
			$d \leftarrow b - z$ \;
			{\bf return} $(c,d)$ \;
			\caption{\texttt{FastTwoSum}} % Algorithm name
			\label{alg:fastTwoSum}   % optional label to refer to
		\end{algorithm}
	\end{minipage}
\end{center}

Fusce varius orci ac magna dapibus porttitor. In tempor leo a neque bibendum sollicitudin. Nulla pretium fermentum nisi, eget sodales magna facilisis eu. Praesent aliquet nulla ut bibendum lacinia. Donec vel mauris vulputate, commodo ligula ut, egestas orci. Suspendisse commodo odio sed hendrerit lobortis. Donec finibus eros erat, vel ornare enim mattis et.

% Numbered question, with an optional title
\begin{question}[\itshape (with optional title)]
	In congue risus leo, in gravida enim viverra id. Donec eros mauris, bibendum vel dui at, tempor commodo augue. In vel lobortis lacus. Nam ornare ullamcorper mauris vel molestie. Maecenas vehicula ornare turpis, vitae fringilla orci consectetur vel. Nam pulvinar justo nec neque egestas tristique. Donec ac dolor at libero congue varius sed vitae lectus. Donec et tristique nulla, sit amet scelerisque orci. Maecenas a vestibulum lectus, vitae gravida nulla. Proin eget volutpat orci. Morbi eu aliquet turpis. Vivamus molestie urna quis tempor tristique. Proin hendrerit sem nec tempor sollicitudin.
\end{question}

Mauris interdum porttitor fringilla. Proin tincidunt sodales leo at ornare. Donec tempus magna non mauris gravida luctus. Cras vitae arcu vitae mauris eleifend scelerisque. Nam sem sapien, vulputate nec felis eu, blandit convallis risus. Pellentesque sollicitudin venenatis tincidunt. In et ipsum libero. Nullam tempor ligula a massa convallis pellentesque.

%----------------------------------------------------------------------------------------
%	PROBLEM 2
%----------------------------------------------------------------------------------------

\section{Implementation}

Proin lobortis efficitur dictum. Pellentesque vitae pharetra eros, quis dignissim magna. Sed tellus leo, semper non vestibulum vel, tincidunt eu mi. Aenean pretium ut velit sed facilisis. Ut placerat urna facilisis dolor suscipit vehicula. Ut ut auctor nunc. Nulla non massa eros. Proin rhoncus arcu odio, eu lobortis metus sollicitudin eu. Duis maximus ex dui, id bibendum diam dignissim id. Aliquam quis lorem lorem. Phasellus sagittis aliquet dolor, vulputate cursus dolor convallis vel. Suspendisse eu tellus feugiat, bibendum lectus quis, fermentum nunc. Nunc euismod condimentum magna nec bibendum. Curabitur elementum nibh eu sem cursus, eu aliquam leo rutrum. Sed bibendum augue sit amet pharetra ullamcorper. Aenean congue sit amet tortor vitae feugiat.

In congue risus leo, in gravida enim viverra id. Donec eros mauris, bibendum vel dui at, tempor commodo augue. In vel lobortis lacus. Nam ornare ullamcorper mauris vel molestie. Maecenas vehicula ornare turpis, vitae fringilla orci consectetur vel. Nam pulvinar justo nec neque egestas tristique. Donec ac dolor at libero congue varius sed vitae lectus. Donec et tristique nulla, sit amet scelerisque orci. Maecenas a vestibulum lectus, vitae gravida nulla. Proin eget volutpat orci. Morbi eu aliquet turpis. Vivamus molestie urna quis tempor tristique. Proin hendrerit sem nec tempor sollicitudin.

% File contents
\begin{file}[hello.py]
\begin{lstlisting}[language=Python]
#! /usr/bin/python

import sys
sys.stdout.write("Hello World!\n")
\end{lstlisting}
\end{file}

Fusce eleifend porttitor arcu, id accumsan elit pharetra eget. Mauris luctus velit sit amet est sodales rhoncus. Donec cursus suscipit justo, sed tristique ipsum fermentum nec. Ut tortor ex, ullamcorper varius congue in, efficitur a tellus. Vivamus ut rutrum nisi. Phasellus sit amet enim efficitur, aliquam nulla id, lacinia mauris. Quisque viverra libero ac magna maximus efficitur. Interdum et malesuada fames ac ante ipsum primis in faucibus. Vestibulum mollis eros in tellus fermentum, vitae tristique justo finibus. Sed quis vehicula nibh. Etiam nulla justo, pellentesque id sapien at, semper aliquam arcu. Integer at commodo arcu. Quisque dapibus ut lacus eget vulputate.

% Command-line "screenshot"
\begin{commandline}
	\begin{verbatim}
		$ chmod +x hello.py
		$ ./hello.py

		Hello World!
	\end{verbatim}
\end{commandline}

Vestibulum sodales orci a nisi interdum tristique. In dictum vehicula dui, eget bibendum purus elementum eu. Pellentesque lobortis mattis mauris, non feugiat dolor vulputate a. Cras porttitor dapibus lacus at pulvinar. Praesent eu nunc et libero porttitor malesuada tempus quis massa. Aenean cursus ipsum a velit ultricies sagittis. Sed non leo ullamcorper, suscipit massa ut, pulvinar erat. Aliquam erat volutpat. Nulla non lacus vitae mi placerat tincidunt et ac diam. Aliquam tincidunt augue sem, ut vestibulum est volutpat eget. Suspendisse potenti. Integer condimentum, risus nec maximus elementum, lacus purus porta arcu, at ultrices diam nisl eget urna. Curabitur sollicitudin diam quis sollicitudin varius. Ut porta erat ornare laoreet euismod. In tincidunt purus dui, nec egestas dui convallis non. In vestibulum ipsum in dictum scelerisque.

% Warning text, with a custom title
\begin{warn}[Notice:]
  In congue risus leo, in gravida enim viverra id. Donec eros mauris, bibendum vel dui at, tempor commodo augue. In vel lobortis lacus. Nam ornare ullamcorper mauris vel molestie. Maecenas vehicula ornare turpis, vitae fringilla orci consectetur vel. Nam pulvinar justo nec neque egestas tristique. Donec ac dolor at libero congue varius sed vitae lectus. Donec et tristique nulla, sit amet scelerisque orci. Maecenas a vestibulum lectus, vitae gravida nulla. Proin eget volutpat orci. Morbi eu aliquet turpis. Vivamus molestie urna quis tempor tristique. Proin hendrerit sem nec tempor sollicitudin.
\end{warn}

%----------------------------------------------------------------------------------------

\printbibliography

%----------------------------------------------------------------------------------------

\section*{Appendices}
\subsection*{Appendix A - Unoptimised Pseudocode Solution}

% TODO

\subsection*{Appendix B - Unoptimised Java Solution}

% TODO

\subsection*{Appendix C - Optimised Pseudocode Solution}

% TODO

\subsection*{Appendix D - Non-Decidable Proof via Pumping Lemma}

We here use the pumping lemma for context-free languages to prove this is not a context-free problem.

Let $L$ be the language described as follows:
A representation of a weighted graph $G$ as a series of vertices followed by a series of weighted edges, using the following syntax:

\begin{itemize}
	\item A representation of a weighted graph $G$ as a series of vertices followed by a series of weighted edges, using the following syntax:
    
    \begin{align*}
		S &= (V), (E), I \\
		V &= V,V \textrm{ } | \textrm{ } I \\
		E &= E,E \textrm{ } | \textrm{ } (I, I, I) \\
		I &= \langle 0-9 \rangle \textrm{ } | \textrm{ } II
    \end{align*}
    
    \begin{itemize}
		\item The graph representation must represent a valid graph - vertices are unique, and edges are only between vertices that exist in the graph (not possible in CFG)
	\end{itemize}

	\item The graph representation is followed by a single number, $k$
	\item $k$ must be greater than or equal to the cost of the minimum spanning tree of $G$
	\item For example: $(1, 2, 3),((1, 3, 2), (1, 2, 2), (2, 3, 3)), 4$ is in the language
\end{itemize}

We assume $L$ is context-free

$\implies\exists\textrm{ } p > 0$ s.t. all strings $s \in L$ can be split into $xyzvw$, s.t.:

\begin{enumerate}
	\item $|yzv| \leq p$
	\item $|yv| > 0$
	\item $\forall i \geq 0, x y^i z v^i w \in L$
\end{enumerate}

We choose $s = (1, 2, 3), ([(1,2,1),]^p (2,3,2)), 3$ where $($ and $)$ are characters in the input alphabet but $[$ and $]$ are not

For simplicity:

Let $A$ be the first sequence of $(1,2,3),($ in $s$

Let $B$ be the middle  $[(1,2,1),]^p$ in $s$

Let $C$ be the final $(2,3,2)), 3$ in $s$

From (1), $yzv$ spans only across at most two adjacent of the three sequences $A$, $B$, and $C$

From (2), $yv$ is not empty, so there are five total cases in which $yzv$ can only span parts of:

\begin{enumerate}
	\item $A$
	\item $A$ and $B$
	\item $B$
	\item $B$ and $C$
	\item $C$
	\end{enumerate}

Pumping $y$ and $v$ to give $s'$ in all cases results in some $s' \notin L$, violating condition (3):

\begin{itemize}
	\item \textit{Cases 1 and 2:} \\
	Pumping anything in $A$ (removal or addition) with $i \neq 1$ will break the syntax of the graph
	\item \textit{Case 3:} \\ 
	Pumping $B$ down to $i=0$ will remove the valid MST, escaping the language
	\item \textit{Cases 4 and 5:} \\
	Pumping anything in $C$ with $i \neq 1$ will break the syntax of the graph
\end{itemize}

$\implies$ The pumping lemma is invalid

$\therefore$ By contradiction, the language is not context-free

$\therefore$ The problem is Tier 3 - Turing-decidable or higher

\end{document}